% Chapter 1

\chapter{Introducci\'on Te\'orica} % Chapter title

\label{ch:introduccion} % For referencing the chapter elsewhere, use \autoref{ch:introduction} 

%----------------------------------------------------------------------------------------



%----------------------------------------------------------------------------------------
\section{BCGs en el Universo Local}
%decir que son, describir sus propiedades mas sobresalientes....
%mencionar que hay indicaciones que algunas propiedades se desvian del comportamiento %gral de las galaxias bcg gigantes comunes

Las BCGs son las galaxias m\'as masivas y luminosas del Universo local. Uicadas en o cerca de los centros de los c\'umulos de galaxias. En primer orden , aparecen como galaxias gigantes el\'ipticas,
sin embargo, como veremos a continuaci\'on, sus propiedades \'unicas y los entornos en los que
viven las diferencian de las gigantes el\'ipticas normales e integran uno de los tipos
de galaxias m\'as interesentantes para entender la historia evolutiva de las galaxias masivas, 
de los c\'umulos de galaxias y de la estructura en gran escala en general.
En esta secc\'io se resumen las propiedades de las BCGs con el fin de evidenciar
cu\'an especiales son.

\subsection{Luminosidad}
Los primeros estudios de las BCGs estuvieron focalizados en sus luminosidades extremadamente
grandes, con magnitudes absolutas en la banda visual $ -23.5 \leq M_{V} \leq -21.5$. T\'ipicamente
las BCGs son 10 veces m\'as luminosas que las galaxias el\'ipticas normales (Sandage $\&$ Hardy 1973;
Schombert 1986), incluso, estudios recientes han demostrado que las luminosidades de las
BCGs son demasiado altas para simplemente caracterizarlas como la parte de la funci\'on de luminosidad est\'andar
(Schechter $\&$ Peebles 1976) de las galaxias el\'ipticas (Tremaine
$\&$ Richstone 1977; Dressler 1978; Bernstein $\&$ Bhavsar 2001, agregarrrrrrrrrrrrrr). Esto implica que 
las BCGs no son simplemente el extremo brillante de las galaxias el\'ipticas normales, si no, pertenecen 
a una clase especial, at\'ipica  y \'unica.
M\'as all\'a de eso, si las BCGs constituyesen el extremo brillante de la funci\'on de luminosidad, la
dispersi\'on en sus luminosidades deber\'ia ser grande (por queeeeeeeeeeeeeeeeee), en aproximadamente 2 magnitudes,
no obstante, mediante estudios en el \'optico e infrarrojo cercano, han demostrado una dispersi\'on intr\'inseca
que no excede las 0.3 magnitudes (Sandage 1988; Aragon-Salamanca, Baugh $\&$ Kauffmann 1998; Collins $\&$ Mann 1998).
La pequen\~na dispersi\'on en las luminosidades de las BCGs, soporta la unicidad de la poblaci\'on
de las BCGs y sugiere que han tenido un proceso evolutivo distinto al de las galaxias el\'ipticas masivas ordinarias.
\subsection{Morfolog\'ia}
La morfolog\'ia de una galaxia es una propiedad muy importante
pues nos otorga pistas sobre los procesos que han estado presentes
durante su foamci\'on y evoluci\'on, de esta manera, las galaxias
m\'as luminosas de los c\'umulos actualmente han adoptado una clasificaci\'on
morfol\'ogica dada por: galasxias \textbf{cDs} y \textbf{BCGs}, la principal
diferencia se debe a la presencia de una gran envolvente en las primeras
que no se presenta en las segundas (figura..). En este trabajo la nomenclatura BCG, abarcar\'a
a toda la poblaci\'on de gigante el\'ipticas luminosas, sin disntinguir por morfolog\'ia.
Lo que es imortante determinar es qu\'e tan distintas son las BCGs respecto a las
galaxias el\'ipticas normales.

\subsection{Estructura}
Puesto que las BCGs parecen tener una estructura que es \'unica a su especie,
fueron muchos los a\'~os dedicados al estudio de \'esta. 
Oelmer (1976) fue el primero en llevar a cabo un trabajo comparativo mediante el
ajustando perfiles de brillo superficial. Encontr\'o que las galaxias el\'ipticas normales
eran bien ajustadas por el modelo utilizado mientras que las BCGs, especialmente las que
poseen la envolvente extensa, se desviaban de los ajustes, adem\'as observ\'o que tales envolventes
generan una especie de inflexi\'on en los perfiles de luz de las BCGs que ocurre
t\'ipicamente en $24 \leq \mu_{v} \leq 26 mag/arcseg^{2}$.
Schombert (1987),
condujo un estudio de los perfiles de luz adoptando como modelo la ley de de Vaucouleurs $r^{1/4}$ (de Vaucouleurs 1948),
asociado a galaxias de tipo temprano. Sus resultaron tambi\'en mostraron diferencias estructurales 
entre las BCGs y las galaxias el\'ipticas normales, sin embargo destacaron que el modelo
s\'olo resultaba bueno, tanto para las el\'ipticas normales como para las BCGs, en un acotado rango de brillos superficiales,
$21 \leq \mu_{v} \leq 25 mag/arcseg^{2}$, siendo necesario as\'i, incurrir en mejores modelos para
estudiar las diferencias estructurales.
Estudios m\'as recientes hacen uso del modelo de S\'ersic basado en la siguiente forma
\begin{equation}
 I(r)=I_{e}exp\{ -b[(r/r_{e})^{1/n}-1] \}
\end{equation}

donde $I(r)$ es la intensidad a una distancia $r$ medida desde el centro, $r_{e}$ es el radio efectivo, definido como
el radio que contiene la mitad de la luminosidad total, $I_{e}$ es la intensidad en $r_{e}$, $n$ es el \'indice de
S\'ersicque representa el grado de condentrtaci\'on y $b \approx 2n-0.33$ (Caon, Capaccioli $\&$ D\' Onofrio 1993) 
Graham et al. (1996) aplic\'o este modelo a los perfiles de luz de las BCGs y encontr\'o que es un modelo
adecuado para representar la estructura la \'estas galaxias. Adem\'as destac\'o que las BCGs presentan \'indices de
S\'ersic m\'as grandes respecto a los asociados con las galaxias el\'ipticas ordinarias. 
No obstante, estudios posteriores demostraron que no era suficiente hacer uso de un \'unico perfil
de S\'ersic para reproducir las distribuci\'on de la luz en las BCGs. Gonzalez, Zabludoff $\&$ Zaritsky (2005)
encontraron que para una muestra de 30 BCGs, ajustar dos perfiles de de Vaoucouleurs daba mejores resultados que
un perfil de S\'ersic pero m\'as tarde,  Donzelli, Muriel $\&$ Madrid (2011) sugirieron que es m\'as apropiado
un modelo basado en dos componentes, una de S\'ersic para la zona interna y una exponencial en la zona m\'as
externa, para descomponer la distruci\'on de la luz de manera adecuada. La interpretaci\'on
de que algunas BCGs no puedan ser modeladas por un \'unico perfil de S\'ersic, suele atribuirse
a que a veces se las encuentra dentro de un halo estelar disperso.
Entonces, dado que las BCGs, a veces se ajustan con un \'unico perfil de S\'ersic, mientras que otras no debido
a la existencia de un halo estelar,
nos conduce a concluir que dentro de la poblaci\'on de BCGs existen dos tipos estructurales de galaxias.
Por ejemplo, Donzelli, Muriel $\&$ Madrid (2011)
tras estudiar separadamente las BCGs cuyos ajustes resultaron favorables con una \'unica componente
(S\'ersic), de aquellas que precisaron dos componentes (S\'ersic+Exponencial), concluyeron
que las BCGs de dos perfiles son m\'as brillantes y que la luz extra que estas poseen 
proviene de regiones que no forman parte de \'estas.
Por lo tanto, estudiar este subconjunto de BCGs puede darnos indicios sobre la evoluci\'on de las BCGs en general.



%----------------------------------------------------------------------------------------

\section{Historia de Formaci\'ion}\label{sec:segunda}
Tomar como guia lo que esta escrito en la introduccion del paper que estamos escribiendo ahora (tenes el link de overleaf). NO delirarse y salir de esa introduccion!!!!!!!!!!!!!!!!!! 

